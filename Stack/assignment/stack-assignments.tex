\documentclass[10pt,aspectratio=169]{beamer}
\usepackage[utf8]{inputenc}
\usepackage[T1]{fontenc}

\usetheme{AnnArbor}
\setbeamercovered{transparent}
\usecolortheme{seahorse}
\setbeamercolor{frametitle}{fg=blue,bg=white}

\definecolor{darkgreen}{rgb}{0,0.4,0}
\definecolor{limegreen}{rgb}{0.196 ,0.809  ,0.196}
\definecolor{magenta}{rgb}{0.99,0,0.99}
\definecolor{maroon}{rgb}{0.687, 0.187, 0.375}
\definecolor{violet}{rgb}{0.539, 0.167, 0.882}
\definecolor{slateblue1}{rgb}{0.513, 0.435, 1}
\definecolor{steelblue1}{rgb}{0.388, 0.721, 1}
\definecolor{gry}{rgb}{0.745 , 0.745 , 0.745}
\definecolor{darkorange}{rgb}{1 , 0.549 , 0}
\definecolor{lavender}{rgb}{0.901, 0.901, 0.980}
\definecolor{lavenderblush4}{rgb}{0.545, 0.513, 0.525}
\definecolor{pblue}{rgb}{0.13,0.13,1}
\definecolor{pgreen}{rgb}{0,0.5,0}
\definecolor{pred}{rgb}{0.9,0,0}
\definecolor{pgrey}{rgb}{0.46,0.45,0.48}
    
\year=2021
\month=11
\day=30

\author{Dr. Matthias Hölzl}

\begin{document}
\title[Stack]{%
  Stack}

\begin{frame}
  \maketitle
\end{frame}

\begin{frame}[fragile]
  \frametitle{Goals}
  \begin{itemize}
  \item Create a simple \texttt{Stack} data structure using TDD
  \item Proceed incrementally, baby steps\\[1ex]
  
  \item This exercise is meant to demonstrate
    \begin{itemize}
    \item how to develop features \texttt{incrementally} using TDD
    \item how to write good tests that help you to grow your design 
    \end{itemize}
  \item Therefore: \textbf{Please do not look ahead in this document!\\
      Only move to the next page after having finished the current
      exercise.}
  \item (Also note that the stack implemented in this workshop has several
    pretty bad design flaws and should not be used for anything but
    demonstrating TDD.)
  \end{itemize}
\end{frame}


\begin{frame}[fragile]
  \frametitle{Stack}
  \begin{itemize}
  \item Open the \texttt{Stack} project in your IDE/editor of choice
  \item Ensure that you can build the target \texttt{StackTest} and
    that the test fails when you execute it
  \end{itemize}
  \bigskip
  \begin{itemize}
  \item \textcolor{darkgreen}{Solve simply!}
  \end{itemize}
\end{frame}

\begin{frame}[fragile]{Stack}
\begin{itemize}
\item Implement a \texttt{Stack} data type for integers with the
  following signature:
\item \texttt{void push(int element)}
\item \texttt{int pop()}
\item \verb!bool is_empty()!
\item If the stack is empty, \texttt{pop()} should throw an exception
  of type \verb!std::out_of_range!
\end{itemize}
\end{frame}

\begin{frame}[fragile]
  \frametitle{Extension (1)}  
  \begin{itemize}
  \item Extend the \texttt{Stack} data type with a method\\[1ex]
    \verb!std::size_t size()!\\[1ex]
    that returns the number of elements on the stack
  \end{itemize}
\end{frame}

\begin{frame}[fragile]
  \frametitle{Extension (2)}  
  \begin{itemize}
  \item Extend the \texttt{Stack} data type with a member function\\[1ex]
    \verb!std::size_t count(int element)!\\[1ex]
    that counts the occurrences of \emph{element} on the stack
  \end{itemize}
\end{frame}

\begin{frame}[fragile]
  \frametitle{Extension (3)}  
  \begin{itemize}
  \item Extend the \texttt{Stack} data type with a member function\\[1ex]
    \verb!int pop_default(int default_value)!\\[1ex]
    that acts like \texttt{pop()} when the stack is not empty and
    returns \verb!default_value! when the stack is empty
  \end{itemize}
\end{frame}

\begin{frame}[fragile]
  \frametitle{Extension (4)}  
  \begin{itemize}
  \item Extend the \texttt{Stack} data type with member functions\\[1ex]
    \verb!void set_default(int default_value)!\\
    \verb!void clear_default()!\\[1ex]
    After \verb!set_default(default_value)! has been called, \texttt{pop()}
    should act like \verb!pop_default(default_value)!. When
    \verb!clear_default()! is called, \texttt{pop()} should revert to
    its original behavior, i.e., throw an exception when the stack is
    empty.
  \end{itemize}
\end{frame}

\begin{frame}[fragile]
  \frametitle{Extension (5)}
  \begin{itemize}
  \item Our \texttt{Stack} class will now be used on an embedded
    system where dynamic memory allocation at runtime is not allowed.
    Therefore we have to replace its implementation with one that does
    not use \verb!std::vector!.

    Modify your implementation so that it uses a
    \verb|std::array<int, 16>| to store its elements. Make
    \texttt{push()} throw an exception of type
    \verb!std::out_of_range! if the stack is full.

    \bigskip
    \emph{Hint:} You will probably need a member variable to indicate
    how many elements are currently stored on the stack, e.g., you could use
    \verb!std::size_t next_index! to indicate the index of the first
    unused element of the stack.
  \end{itemize}
\end{frame}

\end{document}


% \input{body}

%%% Local Variables:
%%% mode: latex
%%% TeX-master: t
%%% End:
