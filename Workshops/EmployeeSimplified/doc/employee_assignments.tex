\documentclass[10pt,aspectratio=169]{beamer}
\usepackage[utf8]{inputenc}
\usepackage[T1]{fontenc}

\usetheme{AnnArbor}
\setbeamercovered{transparent}
% \usecolortheme{seahorse}
\setbeamercolor{frametitle}{fg=blue,bg=white}

\definecolor{darkgreen}{rgb}{0,0.4,0}
\definecolor{limegreen}{rgb}{0.196 ,0.809  ,0.196}
\definecolor{magenta}{rgb}{0.99,0,0.99}
\definecolor{maroon}{rgb}{0.687, 0.187, 0.375}
\definecolor{violet}{rgb}{0.539, 0.167, 0.882}
\definecolor{slateblue1}{rgb}{0.513, 0.435, 1}
\definecolor{steelblue1}{rgb}{0.388, 0.721, 1}
\definecolor{gry}{rgb}{0.745 , 0.745 , 0.745}
\definecolor{darkorange}{rgb}{1 , 0.549 , 0}
\definecolor{lavender}{rgb}{0.901, 0.901, 0.980}
\definecolor{lavenderblush4}{rgb}{0.545, 0.513, 0.525}
\definecolor{pblue}{rgb}{0.13,0.13,1}
\definecolor{pgreen}{rgb}{0,0.5,0}
\definecolor{pred}{rgb}{0.9,0,0}
\definecolor{pgrey}{rgb}{0.46,0.45,0.48}
    
\year=2022
\month=01
\day=18

\author{Dr. Matthias Hölzl}

\begin{document}
\title[Stack]{%
  Employee Workshop\\
  (Simplified Version)}

\begin{frame}
  \maketitle
\end{frame}

\begin{frame}[fragile]
  \frametitle{Goal}
  \Large Make a small program conform to the SOLID and GRASP guidelines,\\[1ex]
  in particular to the SRP.
\end{frame}

\begin{frame}[fragile]
  \frametitle{Notes}
  \begin{itemize}
    \item The code is in \verb|Workshops/EmployeeSimplified|, a possible
      solution is in \verb|Completed/EmployeeSimplifiedCompleted|.
    \item To simplify the code, most classes are defined as structs with
      publicly accessible members. While you should obviously not do this in
      production code, you can follow the same style during the refactorings to
      reduce the amount of boilerplate code you need to write.
    \item Add constructors to your structs, otherwise \verb|std::make_unique()|
      might not work.
    \item You don't have to keep interfaces stable when refactoring, except for
      the interface to the \verb|AugurDB| database.
    \item \textbf{There is no single `correct' solution.} You don't have to
      follow the steps proposed in this outline; they are just there to help you
      if you get stuck. Try to follow your own ideas about refactoring the code
      and see how it turns out.
    \item \textbf{Refactor in small steps. Try to keep your code buildable and
      running as long as possible.}
  \end{itemize}
\end{frame}

\begin{frame}[fragile]
  \frametitle{Step 1: Identify Reasons to Change}
  \begin{itemize}
    \item What reasons to change does the \verb|Employee| class have?
    \item What strategies can you use to resolve them?
    \item Try to refactor the code using the strategy you have identified. If
      you are uncertain whether you're on the right track you can look at the
      following steps to see a possible way forward.
  
  \end{itemize}
\end{frame}

\begin{frame}[fragile]
  \frametitle{Step 2: Extract a \texttt{ReportPrinter} interface}

  The probably easiest class to extract is one that takes care of printing
  Employees. Extract a \texttt{ResourcePrinter} interface and a concrete
  implementation that prints the data to an output stream.
\end{frame}

\begin{frame}[fragile]
  \frametitle{Step 3: Extract a \texttt{PaymentScheme} interface}

  Since payment and work hours are closely related, it might be useful to
  extract them together. Define a \texttt{PaymentScheme} interface that provides
  the necessary member functions to deal with these aspects. 
  
  Implement subclasses for the various kinds of employees that store the
  required data; this allows you to remove a lot of the data from the
  \texttt{Employee} class and clean up the usage of variables for multiple
  unrelated uses.

  To save to the database you will probably need to provide functionaliyt in
  \texttt{PaymentScheme} that supports this. This is not strictly conforming to
  the clean code rules, but it is a good compromise to implement a `quick and
  dirty' solution for this right now, since you need to address saving to the
  database anyway, and any proper solution you implement right now would have to
  be changed anyway.
\end{frame}

\begin{frame}[fragile]
  \frametitle{Step 4: Extract an \texttt{EmployeeDao} interface}

  \begin{itemize}
    \item Extract an interface that can be used to save employees to a database.
    \item Implement the interface for the Augur database.
    \item \textit{Note:} In a realistic scenario you would probably create a
      factory  that creates the correct DAO based on the database you provide.
      To simplify the example, you can simply hard-code the creation of an
      \texttt{AugurDbEmpleyeeDao} whenever you need a DAO.
    \item There is no completely clean way to implement saving the
    \texttt{PaymentScheme} subclasses to the database in C++. You either have to
    add a member function to the \texttt{PaymentScheme} subclasses that provides
    the data in a standardized format, or you have to use a case distinction
    with \verb|dynamic_cast<>()| to identify the subclass of
    \texttt{PaymentScheme} you are currently trying to save.
  \end{itemize} 
\end{frame}

\begin{frame}[fragile]
  \frametitle{Step 5: Admire your clean \texttt{Employee} class}

  Congratulations! You now have an \texttt{Employee} class where all reasons to
  change are delegated to an interface that serves exactly that purpose!
\end{frame}

\end{document}


% \input{body}

%%% Local Variables:
%%% mode: latex
%%% TeX-master: t
%%% End:
