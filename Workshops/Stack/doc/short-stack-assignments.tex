\documentclass[10pt,aspectratio=169]{beamer}
\usepackage[utf8]{inputenc}
\usepackage[T1]{fontenc}

\usetheme{AnnArbor}
\setbeamercovered{transparent}
\usecolortheme{seahorse}
\setbeamercolor{frametitle}{fg=blue,bg=white}

\definecolor{darkgreen}{rgb}{0,0.4,0}
\definecolor{limegreen}{rgb}{0.196 ,0.809  ,0.196}
\definecolor{magenta}{rgb}{0.99,0,0.99}
\definecolor{maroon}{rgb}{0.687, 0.187, 0.375}
\definecolor{violet}{rgb}{0.539, 0.167, 0.882}
\definecolor{slateblue1}{rgb}{0.513, 0.435, 1}
\definecolor{steelblue1}{rgb}{0.388, 0.721, 1}
\definecolor{gry}{rgb}{0.745 , 0.745 , 0.745}
\definecolor{darkorange}{rgb}{1 , 0.549 , 0}
\definecolor{lavender}{rgb}{0.901, 0.901, 0.980}
\definecolor{lavenderblush4}{rgb}{0.545, 0.513, 0.525}
\definecolor{pblue}{rgb}{0.13,0.13,1}
\definecolor{pgreen}{rgb}{0,0.5,0}
\definecolor{pred}{rgb}{0.9,0,0}
\definecolor{pgrey}{rgb}{0.46,0.45,0.48}
    
\year=2022
\month=05
\day=04

\author{Dr. Matthias Hölzl}

\begin{document}
\title[Stack]{%
  Stack\\
  (Short Version)}

\begin{frame}
  \maketitle
\end{frame}

\begin{frame}[fragile]
  \frametitle{Goals}
  \begin{itemize}
    \item Create a simple \texttt{Stack} data structure using TDD
    \item Proceed incrementally, baby steps\\[1ex]

    \item This exercise is meant to demonstrate
          \begin{itemize}
            \item how to develop features \texttt{incrementally} using TDD
            \item how to write good tests that help you to grow your design
          \end{itemize}
    \item Therefore: \textbf{Please do not look ahead in this document!\\
            Only move to the next task after having finished the current one.}
    \item (Also note that the stack implemented in this workshop has several
          pretty bad design flaws and should not be used for anything but
          demonstrating TDD.)
  \end{itemize}
\end{frame}


\begin{frame}[fragile]
  \frametitle{Stack}
  \begin{itemize}
    \item Open the \texttt{Stack} project in your IDE/editor of choice
    \item Ensure that you can build the target \texttt{StackTest} and
          that the test fails when you execute it
  \end{itemize}
  \bigskip
  \begin{itemize}
    \item \textcolor{darkgreen}{Solve simply!}
  \end{itemize}
\end{frame}

\begin{frame}[fragile]{Stack}
  \begin{itemize}
    \item Implement a \texttt{Stack} data type for integers with the
          following signature:
          \begin{itemize}
            \item \texttt{void push(int element) // Add element to the top of the stack}
            \item \texttt{int pop() // Remove the topmost element of the stack and
                    return it }
            \item \verb!bool is_empty() // Check whether the stack is empty!
          \end{itemize}
    \item If the stack is empty, \texttt{pop()} should throw an exception
          of type \verb!std::out_of_range!
  \end{itemize}
  \textbf{Hints:}
  \begin{itemize}
    \item Use a member variable \verb|std::vector<int> elements| to store the
          elements of the vector
    \item You can use the following functions of \verb|std::vector<int>|:
          \begin{itemize}
            \item \verb|elements.push_back(element)| to add \verb|element| to the end of
                  \verb|elements|
            \item \verb|elements.back()| to get the last element in the vector of elements
            \item \verb|elements.pop_back()| to remove the last element in the vector of elements
            \item \verb|elements.empty()| to check whether the vector of elements is
                  empty
          \end{itemize}
  \end{itemize}
\end{frame}

\begin{frame}[fragile]
  \frametitle{Extension (1)}
  \begin{itemize}
    \item Extend the \texttt{Stack} data type with a method\\[1ex]
          \verb!std::size_t size()!\\[1ex]
          that returns the number of elements on the stack
    \item Can you simplify your tests using the \verb|Stack::size()| member
          function? If your tests use implementation details, can you remedy this
          using \verb|Stack::size()|? If so, refactor your tests.
  \end{itemize}
  \textbf{Hint:}
  \begin{itemize}
    \item You can use \verb|elements.size()| to get the number of elements in
          \verb|elements|
  \end{itemize}
\end{frame}

\begin{frame}[fragile]
  \frametitle{Extension (2)}
  \begin{itemize}
    \item Our \texttt{Stack} class will now be used on an embedded system where
          dynamic memory allocation at runtime is not allowed. Therefore we have
          to replace its implementation with one that does not use
          \verb!std::vector!.

          An example implementation is available at \url{https://tinyurl.com/2kcn7v8f}.

          How many of your test stop working if you replace your implementation
          with the one from the link?

          Test that \texttt{push()} throws an exception of type
          \verb!std::out_of_range! if the stack is full.
  \end{itemize}
\end{frame}

\begin{frame}[fragile]
  \frametitle{If you have time left and nothing better to do...}
  \begin{center}
    \LARGE Bonus Content
  \end{center}
\end{frame}

\begin{frame}[fragile]
  \frametitle{Extension (3)}
  \begin{itemize}
    \item Extend the \texttt{Stack} data type with a member function\\[1ex]
          \verb!std::size_t count(int element)!\\[1ex]
          that counts the occurrences of \emph{element} on the stack
  \end{itemize}
\end{frame}

\begin{frame}[fragile]
  \frametitle{Extension (4)}
  \begin{itemize}
    \item Extend the \texttt{Stack} data type with a member function\\[1ex]
          \verb!int pop_default(int default_value)!\\[1ex]
          that acts like \texttt{pop()} when the stack is not empty and
          returns \verb!default_value! when the stack is empty
  \end{itemize}
\end{frame}

\begin{frame}[fragile]
  \frametitle{Extension (5)}
  \begin{itemize}
    \item Extend the \texttt{Stack} data type with member functions\\[1ex]
          \verb!void set_default(int default_value)!\\
          \verb!void clear_default()!\\[1ex]
          After \verb!set_default(default_value)! has been called, \texttt{pop()}
          should act like \verb!pop_default(default_value)!. When
          \verb!clear_default()! is called, \texttt{pop()} should revert to
          its original behavior, i.e., throw an exception when the stack is
          empty.
  \end{itemize}
\end{frame}

\end{document}


% \input{body}

%%% Local Variables:
%%% mode: latex
%%% TeX-master: t
%%% End:
